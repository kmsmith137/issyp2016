\documentclass[aps,prd,superscriptaddress,groupedaddress,nofootinbib,nobibnotes]{revtex4}

\usepackage{graphicx}
\usepackage{dcolumn}
\usepackage{bm}
\usepackage{amssymb}
\usepackage{epstopdf}
\usepackage{amsmath}
\usepackage{amsfonts}
\usepackage{color}
\usepackage{mathrsfs}
% \usepackage{comment}
% \usepackage{url}
% \usepackage{wick}
% \usepackage{feynmp}
% \usepackage{braket}

\setlength{\parindent}{20pt}
% \setlength{\parskip}{1mm}

\setcounter{topnumber}{1}    % default value is 2.
\setcounter{bottomnumber}{0} % default value is 1.

\hyphenation{ALPGEN}
\hyphenation{EVTGEN}
\hyphenation{PYTHIA}

\newcommand{\kms}[1]{\textcolor{blue}{(KMS: #1)}}
\newcommand{\be}{\begin{equation}}
\newcommand{\ee}{\end{equation}}
\newcommand{\ba}{\begin{eqnarray}}
\newcommand{\ea}{\end{eqnarray}}
\newcommand{\nn}{\nonumber}
\newcommand{\barr}{\begin{array}}
\newcommand{\earr}{\end{array}}
\newcommand{\eqdef}{\stackrel{\rm def}{=}}
\newcommand{\bigoh}{\mathcal{O}}

\newcommand\lsim{\mathrel{\rlap{\lower4pt\hbox{\hskip1pt$\sim$}}
        \raise1pt\hbox{$<$}}}
\newcommand\gsim{\mathrel{\rlap{\lower4pt\hbox{\hskip1pt$\sim$}}
        \raise1pt\hbox{$>$}}}

\def\threej#1#2#3#4#5#6{\left( \begin{array}{ccc} #1 & #2 & #3 \\ #4 & #5 & #6 \end{array} \right) }
\def\smallsum{\mathop{\textstyle\sum}\limits}
\def\Var{\mbox{Var}}
\def\Cov{\mbox{Cov}}

\renewcommand{\baselinestretch}{1.1}

\begin{document}

%\title{Some awesome notes}
%\author{Kendrick~M.~Smith}
% \affiliation{Perimeter Institute for Theoretical Physics, Waterloo, ON N2L 2Y5, Canada}

% \date{\today}

% \begin{abstract}
% ABSTRACT HERE
% \end{abstract}
% \pacs{}

% \maketitle

\section{Some math questions}

\begin{enumerate}
\item What is the solution of the system of linear equations
\ba
2x + y &=& 6 \nn \\
x - y &=& -3
\ea
\item A fair coin is flipped four times.  What is the probability of two or more heads?
\item What is the derivative of the function $f(x) = x \sin(2x)$?
\item What is the integral $\int x \sin(2x) \, dx$?
\item What is the general solution $y(x)$ of the second-order differential equation
\be
\frac{d^2y}{dx^2} + \frac{dy}{dx} - 2 y = 0
\ee
\item What is the inverse of the 2-by-2 matrix
\be
\left( \begin{array}{cc}
  2 & 1 \\
  1 & -1
\end{array} \right)
\ee
What are its eigenvalues and eigenvectors?
\item Suppose that a fair coin is flipped $N$ times.  Let $H$ be the total number of heads obtained.
This is an example of a random variable: $H$ can take any value between 0 and $N$,
and we can try to calculate various expectation values involving $H$.  For example, the expectation value 
$\langle H \rangle$ of $H$ itself is clearly equal to $(N/2)$.

The variance of the random variable $H$ is defined by
\be
\Var(H) = \Big\langle (H - \langle H \rangle)^2 \Big\rangle
\ee
The square root $\Var(H)^{1/2}$ is one measure of the ``typical'' size of fluctuations around the mean.
Calculate $\Var(H)$ as a function of $N$.

\item What is the Fourier transform $\tilde f(k) = \int dx \, f(x) e^{ikx}$ of the function $f(x) = 1/(1+x^2)$?
\end{enumerate}

%\begin{figure}
%\centerline{\includegraphics[width=14cm]{x.pdf}}
%\caption{xxx}
%\label{fig:xxx}
%\end{figure}

% \section*{Acknowledgments}
%
% Research at Perimeter Institute is supported by the Government of Canada
% through Industry Canada and by the Province of Ontario through the Ministry of Research \& Innovation.
% Some computations were performed on the GPC cluster at the SciNet HPC Consortium.
% SciNet is funded by the Canada Foundation for Innovation under the auspices of Compute Canada,
% the Government of Ontario, and the University of Toronto.
% KMS was supported by an NSERC Discovery Grant and an Ontario Early Researcher Award.

% \bibliographystyle{h-physrev}
% \bibliography{xxx}

% \appendix
% \section{Appendix}

\end{document}
