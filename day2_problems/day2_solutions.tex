\documentclass[aps,prd,superscriptaddress,groupedaddress,nofootinbib,nobibnotes]{revtex4}

\usepackage{graphicx}
\usepackage{dcolumn}
\usepackage{bm}
\usepackage{amssymb}
\usepackage{epstopdf}
\usepackage{amsmath}
\usepackage{amsfonts}
\usepackage{color}
\usepackage{mathrsfs}
% \usepackage{comment}
% \usepackage{url}
% \usepackage{wick}
% \usepackage{feynmp}
% \usepackage{braket}

\setlength{\parindent}{20pt}
% \setlength{\parskip}{1mm}

\setcounter{topnumber}{1}    % default value is 2.
\setcounter{bottomnumber}{0} % default value is 1.

\hyphenation{ALPGEN}
\hyphenation{EVTGEN}
\hyphenation{PYTHIA}

\newcommand{\kms}[1]{\textcolor{blue}{(KMS: #1)}}
\newcommand{\be}{\begin{equation}}
\newcommand{\ee}{\end{equation}}
\newcommand{\ba}{\begin{eqnarray}}
\newcommand{\ea}{\end{eqnarray}}
\newcommand{\nn}{\nonumber}
\newcommand{\barr}{\begin{array}}
\newcommand{\earr}{\end{array}}
\newcommand{\eqdef}{\stackrel{\rm def}{=}}
\newcommand{\bigoh}{\mathcal{O}}

\newcommand\lsim{\mathrel{\rlap{\lower4pt\hbox{\hskip1pt$\sim$}}
        \raise1pt\hbox{$<$}}}
\newcommand\gsim{\mathrel{\rlap{\lower4pt\hbox{\hskip1pt$\sim$}}
        \raise1pt\hbox{$>$}}}

\def\threej#1#2#3#4#5#6{\left( \begin{array}{ccc} #1 & #2 & #3 \\ #4 & #5 & #6 \end{array} \right) }
\def\smallsum{\mathop{\textstyle\sum}\limits}
\def\Var{\mbox{Var}}
\def\Cov{\mbox{Cov}}

\renewcommand{\baselinestretch}{1.1}

\begin{document}

\title{Setting up differential equations: solutions}
\author{ISSYP 2016}
\maketitle

\par\noindent
Question 5 ``Radioactive decay, part 1'':
\be
\frac{dM}{dt} = -10^{-4} M
\ee
Question 6 ``Radioactive decay, part 2'':
\be
\frac{dA}{dt} = -r_1 A
   \hspace{1.5cm}
\frac{dB}{dt} = r_1 A - r_2 B
   \hspace{1.5cm}
\frac{dC}{dt} = r_2 B
\ee
Question 7 ``Mixing'':
\be
\frac{dM}{dt} = (0.1) (100) - (0.1) \frac{M}{100}
\ee
Question 8 ``Free-fall with air resistance'', solution 1:
\be
\frac{dy}{dt} = v
   \hspace{1.5cm}
\frac{dv}{dt} = 9.8 - kv
\ee
Question 8 ``Free-fall with air resistance'', solution 2.  Compared to solution 1, this just uses the opposite sign convention for $v$ (either is correct):
\be
\frac{dy}{dt} = -v
   \hspace{1.5cm}
\frac{dv}{dt} = -9.8 - kv
\ee
Question 9 ``Roadrunner and coyote'', solution 1.  This uses three dependent variables: the coordinates $(x_c,y_c)$ of the coyote and the $x$-coordinate $x_r$ of the roadrunner.
\ba
\frac{dx_c}{dt} &=& 12 \frac{x_r-x_c}{\sqrt{(x_r-x_c)^2 + y_c^2}} \nn \\
\frac{dy_c}{dt} &=& -12 \frac{y_c}{\sqrt{(x_r-x_c)^2 + y_c^2}} \nn \\
\frac{dx_r}{dt} &=& 10
\ea
Question 9 ``Roadrunner and coyote'', solution 2.  You can also notice that the roadrunner's position $x_r$ is always equal to $10t$, so we don't need to introduce $x_r$ as
a dependent variable.  Instead we get a system of differential equations with two dependent variables:
\ba
\frac{dx_c}{dt} &=& 12 \frac{10t-x_c}{\sqrt{(10t-x_c)^2 + y_c^2}} \nn \\
\frac{dy_c}{dt} &=& -12 \frac{y_c}{\sqrt{(10t-x_c)^2 + y_c^2}}
\ea

%\begin{figure}
%\centerline{\includegraphics[width=14cm]{x.pdf}}
%\caption{xxx}
%\label{fig:xxx}
%\end{figure}

% \section*{Acknowledgments}
%
% Research at Perimeter Institute is supported by the Government of Canada
% through Industry Canada and by the Province of Ontario through the Ministry of Research \& Innovation.
% Some computations were performed on the GPC cluster at the SciNet HPC Consortium.
% SciNet is funded by the Canada Foundation for Innovation under the auspices of Compute Canada,
% the Government of Ontario, and the University of Toronto.
% KMS was supported by an NSERC Discovery Grant and an Ontario Early Researcher Award.

% \bibliographystyle{h-physrev}
% \bibliography{xxx}

% \appendix
% \section{Appendix}

\end{document}
