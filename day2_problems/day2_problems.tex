\documentclass[aps,prd,superscriptaddress,groupedaddress,nofootinbib,nobibnotes]{revtex4}

\usepackage{graphicx}
\usepackage{dcolumn}
\usepackage{bm}
\usepackage{amssymb}
\usepackage{epstopdf}
\usepackage{amsmath}
\usepackage{amsfonts}
\usepackage{color}
\usepackage{mathrsfs}
\usepackage{enumitem}
% \usepackage{comment}
% \usepackage{url}
% \usepackage{wick}
% \usepackage{feynmp}
% \usepackage{braket}

\setlength{\parindent}{20pt}
% \setlength{\parskip}{1mm}

\setcounter{topnumber}{1}    % default value is 2.
\setcounter{bottomnumber}{0} % default value is 1.

\hyphenation{ALPGEN}
\hyphenation{EVTGEN}
\hyphenation{PYTHIA}

\newcommand{\kms}[1]{\textcolor{blue}{(KMS: #1)}}
\newcommand{\be}{\begin{equation}}
\newcommand{\ee}{\end{equation}}
\newcommand{\ba}{\begin{eqnarray}}
\newcommand{\ea}{\end{eqnarray}}
\newcommand{\nn}{\nonumber}
\newcommand{\barr}{\begin{array}}
\newcommand{\earr}{\end{array}}
\newcommand{\eqdef}{\stackrel{\rm def}{=}}
\newcommand{\bigoh}{\mathcal{O}}
\newcommand{\angstrom}{\mbox{\normalfont\AA}}

\newcommand\lsim{\mathrel{\rlap{\lower4pt\hbox{\hskip1pt$\sim$}}
        \raise1pt\hbox{$<$}}}
\newcommand\gsim{\mathrel{\rlap{\lower4pt\hbox{\hskip1pt$\sim$}}
        \raise1pt\hbox{$>$}}}

\def\threej#1#2#3#4#5#6{\left( \begin{array}{ccc} #1 & #2 & #3 \\ #4 & #5 & #6 \end{array} \right) }
\def\smallsum{\mathop{\textstyle\sum}\limits}
\def\Var{\mbox{Var}}
\def\Cov{\mbox{Cov}}

\renewcommand{\baselinestretch}{1.1}

\begin{document}

\title{Problems: Expansion history of the universe and differential equations}

\author{ISSYP 2016}

% \begin{abstract}
% ABSTRACT HERE
% \end{abstract}
% \pacs{}

\maketitle

\par\noindent
Throughout these problems, consider a cosmological expansion history with the following parameters:
\be
H_0 = 0.069 \mbox{Gyr}^{-1}
  \hspace{1cm}
\Omega_m = 0.31
  \hspace{1cm}
\Omega_{\rm rad} = 4.15 \times 10^{-5}
  \hspace{1cm}
\Omega_\Lambda = 1 - \Omega_m - \Omega_{\rm rad}
\ee
Assume that dark energy is a cosmological constant.

\section{Expanding universe}

\begin{enumerate}
\item Using Friedmann's equation and the definition of $\Omega_i$, calculate numerical
  values for the density $\rho_{\rm tot}$ of the universe today, the matter density $\rho_m$
  today, and the radiation density $\rho_{\rm rad}$ today.
\item Find expressions for the matter density $\rho_m(a)$, the radition density $\rho_{\rm rad}(a)$,
  the total density $\rho_{\rm tot}(a)$, and the Hubble parameter $H(a)$, as functions of $a$.  
  Note that we're using $a$ as the time coordinate here, i.e.~all quantities are evaluated at 
  a time when the scale factor of the universe is $a$.
\item Galaxies emit light in a narrow spectral line with wavelength 1216 \AA (the ``Lyman alpha'' line).
  Consider a distant galaxy which is redshifted so that the line is observed at 2800 \AA.
  How much has the universe expanded between the time when the light was emitted and now?
  How much has the total density of the universe $\rho_{\rm tot}$ decreased since then?
\item What was the scale factor of the universe when the densities of matter and dark energy were equal?
  When the densities of matter and radiation were equal?
\end{enumerate}
  
\section{Setting up differential equations}

\begin{enumerate}[resume]
\item {\em Radioactive decay, part 1.} Consider a radioactive sample which decays
  at the rate $r = 0.0001$ yr$^{-1}$, meaning that in one year, 0.01\% of the sample decays.
  Write a differential equation for the mass of the sample as a function of time.

\item {\em Radioactive decay, part 2.} Now consider a ``decay chain''.  Suppose that radioactive
  element $A$ decays to element $B$ with rate $r_1 = 0.0001$ yr$^{-1}$, and element $B$ decays in
  turn to element $C$ with rate $r_2 = 0.0003$ yr$^{-1}$.  Write differential equations for the
  masses of samples $A$, $B$, and $C$ as functions of time.

\item {\em Mixing.} Consider a tank which initially contains 100 liters of pure water.
  At time $t=0$, an input valve is turned on which feeds salt water into the tank.
  Assume that the salt water instantaneously mixes with the water in the tank, so that the
  concentration of salt is always uniform throughout the tank.
  To keep the total volume of water in the talk constant at 100 liters, an output valve
  is also turned on (at $t=0$) which drains water from the tank at the same rate.
  Suppose that the rate of water in (and out) of the tank is 0.1 liters per second, and
  the concentration of salt in the input stream is 100 grams per liter.
  Write a differential equation for the total mass of salt $M$ in the tank as a function of time.

\item {\em Free-fall with air resistance.} A brick of mass $m = 2$ kg is dropped out of an airplane at 
  altitude 10000 m.  Recall that the acceleration of the brick due to gravity is 9.8 $m/s^2$.  Assume
  that the effect of air resistance is to supply a drag force proportional to velocity: $F = kv$,
  where $k = 0.1$ kg/s.
  Find differential equations which can be solved to find the location of the brick at time $t$.

\item {\em Roadrunner and coyote.} Consider a ``roadrunner'' who starts at the origin (0,0) of the
  $(x,y)$ plane, and runs in the $x$-direction at constant velocity 10 $m/s$.  The roadrunner is
  being chased by a ``coyote'', who starts at the point (0,100) and runs with velocity 12 $m/s$.
  Assume that the coyote always runs directly toward the roadrunner's instantaneous position.
  Find differential equations which could be solved to determine how long it takes the roadrunner
  to catch the coyote.
\end{enumerate}

\section{Using python}

\begin{enumerate}[resume]

\item Make a length-50 list whose $n$-th element (for $0 \le n < 50$) is equal to $(1.1)^n + 5n^2$.

\item The Fibonacci numbers $F_0=1$, $F_1=1$, $F_2=2$, $F_3=3$, $F_4=5$, $F_5=8$, $\cdots$
  are defined by the recursion relation $F_n = F_{n-1} + F_{n-2}$.  Write a python program to calculate $F_{50}$.

\item Make a list of 100 numbers $t_i$ uniformly spaced between 0 and $(2\pi)$.
  Make another length-100 list containing the values of $\cos(t_i)$.  By calling
  {\tt matplotlib.pyplot.plt()} with these lists as arguments, make a plot of $t$ vs $\cos(t)$.

\item Choose one of the differential equations you derived in the last section, and
  write a short python program to integrate it and plot the result.

\end{enumerate}

\section{Integrating Friedmann's equation with Python}

Warning: everything in this section is supposed to be a challenge!
Don't worry if you find it rough going at first.

Friedmann's equation is the differential equation:
\be
\frac{da}{dt} = a H(a)
\ee
where $H(a)$ has a somewhat messy algebraic form that you derived in question 2 above.

For numerically integrating this equation I suggest making time be the {\em dependent variable}
and using $\log a$ (not $a$!) as the independent variable.  In other words, your numerical integration
would take equal steps in $\log a$, and evolve the value of $t$ (the elapsed time since the big bang).

Something which isn't supposed to be obvious, but which you can show 
\be
\frac{dt}{d\log a} = \frac{1}{H(a)}
\ee
or equivalently, defining $u=\log(a)$ for notational clarity
\be
\frac{dt}{du} = \frac{1}{H(e^u)}
\ee
Write a python program to integrate this differential equation and tabulate $t$ as a function of $u$.
Based on your tabulated values, answer the following questions:

\begin{enumerate}[resume]
\item What is the age of the universe?
\item Consider the redshifted galaxy from question 3 above.  How long ago was the
  light from the galaxy emitted?
\item How different would the age of the universe be if there were no dark energy, but
   the matter and radiation densities $\rho_m$ and $\rho_{\rm rad}$ were the same?
\item Determine whether the expansion of the universe is accelerating or decelerating today,
  by determining whether $da/dt$ is increasing or decreasing with the exapansion.  Would
  the answer be different if there were no dark energy?
\item Warning: this one is especially difficult, I'll give some hints tomorrow!
  What is the comoving size of the universe?  I.e.~if we consider the furthest point we
  can look back to, how far away is that point today?
\end{enumerate}

%\begin{figure}
%\centerline{\includegraphics[width=14cm]{x.pdf}}
%\caption{xxx}
%\label{fig:xxx}
%\end{figure}

% \section*{Acknowledgments}
%
% Research at Perimeter Institute is supported by the Government of Canada
% through Industry Canada and by the Province of Ontario through the Ministry of Research \& Innovation.
% Some computations were performed on the GPC cluster at the SciNet HPC Consortium.
% SciNet is funded by the Canada Foundation for Innovation under the auspices of Compute Canada,
% the Government of Ontario, and the University of Toronto.
% KMS was supported by an NSERC Discovery Grant and an Ontario Early Researcher Award.

% \bibliographystyle{h-physrev}
% \bibliography{xxx}

% \appendix
% \section{Appendix}

\end{document}
