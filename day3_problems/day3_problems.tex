\documentclass[aps,prd,superscriptaddress,groupedaddress,nofootinbib,nobibnotes]{revtex4}

\usepackage{graphicx}
\usepackage{dcolumn}
\usepackage{bm}
\usepackage{amssymb}
\usepackage{epstopdf}
\usepackage{amsmath}
\usepackage{amsfonts}
\usepackage{color}
\usepackage{mathrsfs}
\usepackage{enumitem}
% \usepackage{comment}
% \usepackage{url}
% \usepackage{wick}
% \usepackage{feynmp}
% \usepackage{braket}

\setlength{\parindent}{20pt}
% \setlength{\parskip}{1mm}

\setcounter{topnumber}{1}    % default value is 2.
\setcounter{bottomnumber}{0} % default value is 1.

\hyphenation{ALPGEN}
\hyphenation{EVTGEN}
\hyphenation{PYTHIA}

\newcommand{\kms}[1]{\textcolor{blue}{(KMS: #1)}}
\newcommand{\be}{\begin{equation}}
\newcommand{\ee}{\end{equation}}
\newcommand{\ba}{\begin{eqnarray}}
\newcommand{\ea}{\end{eqnarray}}
\newcommand{\nn}{\nonumber}
\newcommand{\barr}{\begin{array}}
\newcommand{\earr}{\end{array}}
\newcommand{\eqdef}{\stackrel{\rm def}{=}}
\newcommand{\bigoh}{\mathcal{O}}
\newcommand{\angstrom}{\mbox{\normalfont\AA}}

\newcommand\lsim{\mathrel{\rlap{\lower4pt\hbox{\hskip1pt$\sim$}}
        \raise1pt\hbox{$<$}}}
\newcommand\gsim{\mathrel{\rlap{\lower4pt\hbox{\hskip1pt$\sim$}}
        \raise1pt\hbox{$>$}}}

\def\threej#1#2#3#4#5#6{\left( \begin{array}{ccc} #1 & #2 & #3 \\ #4 & #5 & #6 \end{array} \right) }
\def\smallsum{\mathop{\textstyle\sum}\limits}
\def\Var{\mbox{Var}}
\def\Cov{\mbox{Cov}}

\renewcommand{\baselinestretch}{1.1}

\begin{document}

\title{Problems: Python, and integrating the expansion history of the universe}

\author{ISSYP 2016}

% \begin{abstract}
% ABSTRACT HERE
% \end{abstract}
% \pacs{}

\maketitle

\section{Python loops}

\begin{enumerate}

\item Write a program which prints the values of $(1/n^2)$ for $n=1,\cdots,20$.

\item There is a math identity (this is not supposed to be obvious, the proof is hard!) which says
that the infinite series
\be
\frac{1}{1^2} + \frac{1}{2^2} + \frac{1}{3^2} + \cdots
\ee
is equal to $(\pi^2/6)$.  Let's check this result numerically as follows.  Write a loop which calculates the sum to some
maximum value of $n$, and evaluate the result for $n_{\rm max}$ equal to 10, 100, 1000, and 10000.
You should be able to see emperically that the result is converging to $\pi^2/^6$.

\item In one of the problems from yesterday (number 3), we figured out how to compute the total density of the universe
$\rho_{\rm tot}$ at a specific value of $a$ (namely $a=0.43$).  Write a program which prints the density of the universe
at $a = 0.1, 0.2, \cdots, 0.9, 1$.

\item (Hard) The Fibonacci numbers $F_0=1$, $F_1=1$, $F_2=2$, $F_3=3$, $F_4=5$, $F_5=8$, $\cdots$
  are defined by the recursion relation $F_n = F_{n-1} + F_{n-2}$.  Write a python program to print the value of $F_{50}$.
  Can you modify your program to also print the value of $F_0 + F_1 + \cdots + F_{50}$?

\end{enumerate}

\section{Python lists}

\begin{enumerate}[resume]

\item Make a length-50 list whose $n$-th element (for $0 \le n < 50$) is equal to $(1.1)^n + 5n^2$.

\item Make a list of 100 numbers $t_i$ uniformly spaced between 0 and $(2\pi)$.
  Make another length-100 list containing the values of $\cos(t_i)$.  By calling
  {\tt matplotlib.pyplot.plt()} with these lists as arguments, make a plot of $t$ vs $\cos(t)$.

\end{enumerate}

\section{Integrating Friedmann's equation with Python}

\par\noindent
Warning: everything in this section is supposed to be a challenge!
Don't worry if you find it tough at first.

Friedmann's equation is the differential equation:
\be
\frac{da}{dt} = a H(a)
\ee
where $H(a)$ has a somewhat messy algebraic form that you derived in question 2 above.

For numerically integrating this equation I suggest making time be the {\em dependent variable}
and using $\log a$ (not $a$!) as the independent variable.  In other words, your numerical integration
would take equal steps in $\log a$, and evolve the value of $t$ (the elapsed time since the big bang).

Something which isn't supposed to be obvious, but which you can show if you've taken some calculus,
is that the form of Friedmann's equations in these variables is:
\be
\frac{dt}{d\log a} = \frac{1}{H(a)}
\ee
or equivalently, defining $u=\log(a)$ for notational clarity
\be
\frac{dt}{du} = \frac{1}{H(e^u)}
\ee
Write a python program to integrate this differential equation and tabulate $t$ as a function of $u$.
Based on your tabulated values, answer the following questions:

\begin{enumerate}[resume]
\item What is the age of the universe?
\item Consider the redshifted galaxy from question 3 above.  How long ago was the
  light from the galaxy emitted?
\item How different would the age of the universe be if there were no dark energy, but
   the matter and radiation densities $\rho_m$ and $\rho_{\rm rad}$ were the same?
\item Determine whether the expansion of the universe is accelerating or decelerating today,
  by determining whether $da/dt$ is increasing or decreasing with the exapansion.  Would
  the answer be different if there were no dark energy?
\item Warning: this one is particularly difficult, I'll give some hints on Monday!
  What is the comoving size of the universe?  I.e.~if we consider the furthest point we
  can look back to, how far away is that point today?
\end{enumerate}

%\begin{figure}
%\centerline{\includegraphics[width=14cm]{x.pdf}}
%\caption{xxx}
%\label{fig:xxx}
%\end{figure}

% \section*{Acknowledgments}
%
% Research at Perimeter Institute is supported by the Government of Canada
% through Industry Canada and by the Province of Ontario through the Ministry of Research \& Innovation.
% Some computations were performed on the GPC cluster at the SciNet HPC Consortium.
% SciNet is funded by the Canada Foundation for Innovation under the auspices of Compute Canada,
% the Government of Ontario, and the University of Toronto.
% KMS was supported by an NSERC Discovery Grant and an Ontario Early Researcher Award.

% \bibliographystyle{h-physrev}
% \bibliography{xxx}

% \appendix
% \section{Appendix}

\end{document}
