\documentclass[aps,prd,superscriptaddress,groupedaddress,nofootinbib,nobibnotes]{revtex4}

\usepackage{graphicx}
\usepackage{dcolumn}
\usepackage{bm}
\usepackage{amssymb}
\usepackage{epstopdf}
\usepackage{amsmath}
\usepackage{amsfonts}
\usepackage{color}
\usepackage{mathrsfs}
% \usepackage{comment}
% \usepackage{url}
% \usepackage{wick}
% \usepackage{feynmp}
% \usepackage{braket}

\setlength{\parindent}{20pt}
% \setlength{\parskip}{1mm}

\setcounter{topnumber}{1}    % default value is 2.
\setcounter{bottomnumber}{0} % default value is 1.

\hyphenation{ALPGEN}
\hyphenation{EVTGEN}
\hyphenation{PYTHIA}

\newcommand{\kms}[1]{\textcolor{blue}{(KMS: #1)}}
\newcommand{\be}{\begin{equation}}
\newcommand{\ee}{\end{equation}}
\newcommand{\ba}{\begin{eqnarray}}
\newcommand{\ea}{\end{eqnarray}}
\newcommand{\nn}{\nonumber}
\newcommand{\barr}{\begin{array}}
\newcommand{\earr}{\end{array}}
\newcommand{\eqdef}{\stackrel{\rm def}{=}}
\newcommand{\bigoh}{\mathcal{O}}

\newcommand\lsim{\mathrel{\rlap{\lower4pt\hbox{\hskip1pt$\sim$}}
        \raise1pt\hbox{$<$}}}
\newcommand\gsim{\mathrel{\rlap{\lower4pt\hbox{\hskip1pt$\sim$}}
        \raise1pt\hbox{$>$}}}

\def\threej#1#2#3#4#5#6{\left( \begin{array}{ccc} #1 & #2 & #3 \\ #4 & #5 & #6 \end{array} \right) }
\def\smallsum{\mathop{\textstyle\sum}\limits}
\def\Var{\mbox{Var}}
\def\Cov{\mbox{Cov}}

\renewcommand{\baselinestretch}{1.1}

\begin{document}

\title{Problems: Gaussian random variables and a little bit of linear algebra}

\author{ISSYP 2016}
% \affiliation{Perimeter Institute for Theoretical Physics, Waterloo, ON N2L 2Y5, Canada}
% \date{\today}

% \begin{abstract}
% ABSTRACT HERE
% \end{abstract}
% \pacs{}

\maketitle

\begin{enumerate}

\item {\em Simulating Gaussian random variables.}
A question which sometimes arises is how to simulate a Gaussian random variable in $N$ dimensions,
assuming the existence of an algorithm for randomly simulating a one-dimensional Gaussian.\footnote{All
modern programming lanugages have this built in.  For example, in python you can get a Gaussian random number
with {\tt numpy.random.standard\_normal()}.}  In this problem we will study this question in the case $N=2$.
\par\medskip
(a) 
Consider a 2-component Gaussian random variable $X_i$, with mean $\bar X_i = 0$ assumed for simplicity,
and covariance matrix
\be
C_{ij} = \langle X_i X_j \rangle = \left( \begin{array}{cc}
 C_{11} & C_{12} \\
 C_{12} & C_{22}
\end{array} \right)
\ee
Suppose we simulate independent one-dimensional Gaussian random numbers $u_1, u_2$ with variance $\sigma^2=1$.
Show that if we define:
\ba
X_1 &=& \sqrt{C_{11}} u_1 \nn \\
X_2 &=& \frac{C_{12}}{\sqrt{C_{11}}} u_1 + \sqrt{ C_{22} - \frac{C_{12}^2}{C_{11}}} u_2  \label{eq:msqrt}
\ea
then $\langle X_i X_j \rangle = C_{ij}$.  This is one algorithm for simulating a two-dimensional Gaussian.
\par\medskip
(b) More generally, show that a linear transformation of the form $X_i = A_{ij} u_j$ gives covariance
matrix $\langle X_i X_j \rangle = C_{ij}$ if and only if $C_{ij} = A_{ik} A_{jk}$.\footnote{If this equation
holds then we say that $A$ is a ``matrix square root'' of $C$.  Thus the problem of simulating multivariate
Gaussian random numbers is linked to the problem of finding matrix square roots (which turn out to be non-unique).  
The coefficients given in Eq.~(\ref{eq:msqrt}) are one choice of matrix square root in the 2-by-2 case.}
\par\medskip
(c) If you're comfortable with computer programming then try coding this up!  For a specific choice of
covariance matrix, say $C_{11}=2$, $C_{12}=1$, and $C_{22}=3$, try making random realizations
of $X_i$ using Eq.~(\ref{eq:msqrt}), and verify that the averages $\langle X_i X_j \rangle$ converge to $C_{ij}$
in the limit of many realizations.

\item {\em 2-by-2 matrix inversion.} 
Consider a general 2-by-2 matrix
\be
A = \left( \begin{array}{cc}
  A_{11} & A_{12} \\
  A_{21} & A_{22}
\end{array} \right)
\ee
Show that the matrix elements of the inverse matrix
\be
A^{-1} = \left( \begin{array}{cc}
  A^{-1}_{11} & A^{-1}_{12} \\
  A^{-1}_{21} & A^{-1}_{22}
\end{array} \right)
\ee
are given explicitly by
\ba
A^{-1}_{11} &=& \frac{A_{22}}{A_{11}A_{22} - A_{12} A_{21}} \nn \\
A^{-1}_{12} &=& -\frac{A_{12}}{A_{11}A_{22} - A_{12} A_{21}} \nn \\
A^{-1}_{21} &=& -\frac{A_{21}}{A_{11}A_{22} - A_{12} A_{21}} \nn \\
A^{-1}_{22} &=& \frac{A_{11}}{A_{11}A_{22} - A_{12} A_{21}}
\ea
A brute force approach is to show that the defining condition for the inverse matrix,
namely $A_{ij} A^{-1}_{jk} = \delta_{ik}$, holds for all four choices of indices $i,k$.
If you have taken some linear algebra, you may know some machinery which is more efficient
than the brute force approach -- if so then feel free to use it!

\item {\em Gaussian integrals.} In this problem, you can assume that:
\be
\int_{-\infty}^\infty dx \, e^{-x^2} = \sqrt{\pi}
\ee
(If you're curious how this is shown, there is a famous trick which is explained in the appendix!)
\par\medskip
(a) By change of variable show that
\be
\int_{-\infty}^\infty dx \, e^{-ax^2} = \sqrt{\pi} a^{-1/2}  \label{eq:gaussian_a}
\ee
where $a > 0$ is a real number.
\par\medskip
(b) Now show that 
\be
\int_{-\infty}^\infty dx \, x^2 e^{-ax^2} = \frac{\sqrt{\pi}}{2} a^{-3/2}   \label{eq:gaussian_a2}
\ee
using one of two possible approaches.  One way is to group the integrand on the LHS as $(x) (x e^{-ax^2})$
and use integration by parts.  The second way is to differentiate both sides of Eq.~(\ref{eq:gaussian_a})
with respect to $a$.
\par\medskip
(c) Using the previous results show that the Gaussian probability distribution in one variable
\be
p(x) = \frac{1}{(2\pi)^{1/2} \sigma} \exp\left( - \frac{x^2}{2\sigma^2} \right)
\ee
is correctly normalized (i.e.~$\int p(x) = 1$) with variance $\sigma^2$ 
(i.e.~$\int x^2 p(x) = \sigma^2$), as implicitly assumed in the lecture.
\par\medskip
(d) Can you generalize part (b) to give a formula for $\int x^N e^{-ax^2}$, where $N$ is a positive integer?

\end{enumerate}


%\begin{figure}
%\centerline{\includegraphics[width=14cm]{x.pdf}}
%\caption{xxx}
%\label{fig:xxx}
%\end{figure}

% \section*{Acknowledgments}
%
% Research at Perimeter Institute is supported by the Government of Canada
% through Industry Canada and by the Province of Ontario through the Ministry of Research \& Innovation.
% Some computations were performed on the GPC cluster at the SciNet HPC Consortium.
% SciNet is funded by the Canada Foundation for Innovation under the auspices of Compute Canada,
% the Government of Ontario, and the University of Toronto.
% KMS was supported by an NSERC Discovery Grant and an Ontario Early Researcher Award.

% \bibliographystyle{h-physrev}
% \bibliography{xxx}

\appendix
\section{A famous trick for calculating $\int_{-\infty}^\infty dx e^{-x^2}$}

\par\noindent
As far as I know, the following strange trick is the only way of doing the integral!
You'll need to have studied a little bit of multivariate calculus, in particular changing variables from
Cartesian to polar coordinates in a 2D integral.
Define:
\be
I = \int_{-\infty}^\infty dx \, e^{-x^2}
\ee
We can write $I^2$ as a 2D integral over the $(x,y)$ plane.
\ba
I^2 &=& \left( \int_{-\infty}^\infty dx \, e^{-x^2} \right) \left( \int_{-\infty}^\infty dy \, e^{-y^2} \right)  \nn \\
  &=& \int \int dx\, dy \, e^{-x^2-y^2}
\ea
This doesn't appear to be making progress, but if we now change variables to polar coordinates $(r,\theta)$,
then we can do the integral!  (In the steps below, we integrate over $\theta$ first and then $r$.)
\ba
I^2 &=& \int dr \, d\theta \, r e^{-r^2}  \nn \\
    &=& 2 \pi \int_0^\infty dr \, r e^{-r^2} \nn \\
    &=& 2 \pi \left( \frac{1}{2} \right)
\ea
Note that the extra factor of $r$ we picked up in the change of variables to polar coordinates is what
allows us to do the integral.  Taking the square root on both sides we now get $I = \sqrt{\pi}$ as desired.

\end{document}
